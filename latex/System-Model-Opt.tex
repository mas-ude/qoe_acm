%!TEX root = main.tex
%%%%%%%%%%%%%%%%%%%%%%%%%%%%%%%%%%%%%%%%%%%%%%%%%%%%%%%%%%%%%%%%%%%%%%%%%%%%%%%%
\subsection{Optimization Problem}

In order to determine how much potential there is for optimization, we determine the optimal solution for the "adaptation" \textit{(change expression)} by establishing an MILP for this problem. Assuming that we formulate it correctly, the solution to the MILP will return an optimal "adaptation".

A given video is available in $r$ resolutions and consists of $n$ segments, i.e. each segment can be replayed in exactly one resolution. Furthermore, each segment $i$ that is replayed in resolution $j$ has a size $S_{ij}$. We assume that all segments have the same length $\tau$ and are downloaded in order. The total data that has been downloaded at the point in time $t$ is $V(t)$. Before a segment can be replayed, it has to be downloaded. This means there is a deadline $D_i$ until which the segment must be downloaded if no stalling may occur. Since there is an initial delay $T_0$ before the first segment can be replayed, according to \cite{hossfeld2015identifying} the deadline is

\begin{equation}
D_i = T_0 + i\cdot \tau.
\end{equation}

The goal is to optimize the downloading process so that the video may be replayed with the highest average resolution. This leads us to a MILP which is a modified version of the MILP given in \cite{hossfeld2015identifying}.

\begin{align*}
& \text{maximize} & \sum_{i = 1}^{n} \sum_{j = 1}^{r_{\text{max}}} j x_{ij} &\\
& \text{subject to} & &&\\
&& x_{ij} &\in \{0,1\} &\\
&& \sum_{j = 1}^{r_{\text{max}}} x_{ij} &= 1, &\forall i&=1,\ldots,n \\
&& \phantom{\text{.}} \sum_{i=1}^{k} \sum_{j = 1}^{r_{\text{max}}} S_{ij} x_{ij} &\leq V(D_k), &\forall k&=1,\ldots,n \text{.} \\
\end{align*}

\textit{What is the MILP good for?}