%!TEX root = main.tex
%%%%%%%%%%%%%%%%%%%%%%%%%%%%%%%%%%%%%%%%%%%%%%%%%%%%%%%%%%%%%%%%%%%%%%%%%%%%%%%%

\subsection{Data Sets}
\label{sec:datasets}

In total, there are four data sets used in this evaluation as listed in Table \ref{tab:datasets}. The three data-sets starting from the second are calculated based on the first (experimental) data set.
\begin{table}
\caption{Overview of the data sets used in this work.}
\label{tab:datasets}
\centering
\begin{tabular}{p{1.8cm}p{2.1cm}p{3.2cm}}
%\begin{tabular}{|c|c|c|}
\toprule
Data Set & Identifier & Description \\ 
\midrule 
Measurements from \cite{sieber16sacrificing} & measurement & The experimental data set recorded in a testbed. \\ 
\addlinespace
Heuristic estimation & heuristic & The heuristic estimation which gives us the possible gain without redundant traffic. \\ 
\addlinespace
Optimization with stalling & opt (prebuffering) & The MILP solition with stalling times. \\ 
\addlinespace
Opimtization without stalling & opt (instant play) & The MILP solution without stalling times. \\ 
\bottomrule
\end{tabular} 
\end{table}
%Two of those are data sets from \cite{sieber16sacrificing}. The other two were created by implementing above optimization problem in the Gurobi Optimizer

First, we have the initial observations which shall serve as the \textit{baseline} in the following analysis. These measurements were originally recorded in \cite{sieber16sacrificing} where the measurement methodology and measurement set-up is described in more detail: $35$ videos $\times 27$ bandwidth values $\times 15$ replications. Four quality level representations were observed: $144p, 240p, 360p, 480p$. In the following, we refer to these video quality levels as $0,1,2,3$. Please note that stalling events did occur in $56\%$ of these runs.

Based on this data set, we used the \textit{heuristic approach} described in \ref{sec:heu} to estimate the average resolution that is reachable if there was no redundant traffic, i.e. when no video segment is downloaded multiple times. Please note that it was assumed that the same amount of stalling would occur.

As a new contribution, we use the optimization problem, described in Section \ref{optadapt} to exactly calculate the highest mean resolution that was optimally obtainable. As a second step, the number of switches is minimized as first proposed in \cite{miller2013optimal}. For both steps, we limit the execution time of the Gurobi Optimizer to $\SI{1}{\second}$ in order to process the complete data in a timely manner. Increasing the execution will most likely lead to slightly better values than presented in the following. For this \textit{two-step approach}, we consider the same video files, the same duration of the viewing session and the same average network throughput as was used in the baseline scenario to make it comparable. However, instead of having stalling events interrupt the replaying process, we add an initial delay to the replaying process. The duration of this delay is equal to the sum of the observed stalling events. This leads to the same duration of the viewing session and the same replay time and the same amount of data that was totally downloaded. In the following, we refer to this scenario as \textit{opt (prebuffering)}.

Lastly, we present a data set that is obtained in the same fashion as \textit{opt (prebuffering)} with one major difference: the video starts to play immediately after the first segment has been downloaded. To achieve this, we consider the exact same network throughput as in the baseline scenario, while having a shorter session duration since the stalling times are omitted. This means that the amount of data that is downloaded in this case is lower than in the baseline scenario. In the following, we refer to this scenario as \textit{opt (instant play)}.
