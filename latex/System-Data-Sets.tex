%!TEX root = main.tex
%%%%%%%%%%%%%%%%%%%%%%%%%%%%%%%%%%%%%%%%%%%%%%%%%%%%%%%%%%%%%%%%%%%%%%%%%%%%%%%%

\subsection{Data Sets}

In total, we acquired four data that are used in the evaluation. Two of those are data sets from \cite{sieber16sacrificing}. The other two were created by implementing above optimization problem in the Gurobi Optimizer\footnote{http://www.gurobi.com/}.

First, we have the initial observations which shall serve as the \textit{baseline} in the following analysis. Please notice that stalling events did occur during these runs. These measurements were originally recorded in \cite{sieber16sacrificing} where the measurement methodology and measurement set-up is described in great detail: $35$ videos $\times 27$ bandwidth values $\times 15$ replications. Four quality level representations were observed: $144p, 240p, 360p, 480p$. In the following, we refer to these video quality levels as $0,1,2,3$.

Based on this data set, a \textit{heuristic} was developed in \cite{sieber16sacrificing} which estimated the average resolution that is reachable if there was no redundant traffic, i.e. when no video segment is downloaded multiple times. Notice that it was assumed that the same amount of stalling would occur.

As a new contribution, we use the optimization problem, described in Section \ref{optadapt} to exactly calculate the highest mean resolution that was optimally obtainable. As a second step, the number of switches is minimized as first proposed in \cite{miller2013optimal}. For both steps, we limited the execution time of the Gurobi Optimizer to $\SI{1}{second}$ in order to process the complete data in a timely manner. Increasing the execution will most likely lead to slightly better values than presented in the following. For this two-step approach, we consider the same video files, the same duration of the viewing session and the same average network throughput as was used in the baseline scenario to make it comparable. However, instead of having stalling events interrupt the replaying process, we added an initial delay to the the replaying process. The duration of this delay is equal to the sum of the observed stalling events. This leads to the same duration of the viewing session and the same replay time and the same amount of data that was totally downloaded. In the following, we refer to this scenario as \textit{optimization stalling}.

Lastly, we present a data set that is obtained in the same fashion as \textit{optimization stalling} with one mayor difference: the video started replaying as soon as the first segment was fully downloaded. To achieve this, we considered the exact same network throughput as in the baseline scenario, while having a shorter session duration since the stalling times are omitted. This means that the amount of data that is downloaded in this case is lower than in the baseline scenario. In the following, we refer to this scenario as \textit{optimization, no stalling}.
