%!TEX root = main.tex
%%%%%%%%%%%%%%%%%%%%%%%%%%%%%%%%%%%%%%%%%%%%%%%%%%%%%%%%%%%%%%%%%%%%%%%%%%%%%%%%

\section{Methodology}

\label{sec:sysmodel}

In this section we discuss the two approaches we use to evaluate the observed adaptation from the experimental data set.
This first approach is based on regression and uses previously observed video sessions to create an estimation on how much non-redundant traffic relates to a specific average playback quality.
This has the advantage of being fast, scalable and not computational expensive. 
Furthermore, as it is based on actual observed data, it captures the dynamics of the deployed system.
We use this estimation then to calculate the maximum achievable average quality level based on the total amount of downloaded Bytes in a playback session.
The second approach is based on a mixed integer linear program (MILP) formulation. 
For this optimization problem we take the actual video segment sizes, the observed bandwidth and cumulative stallings times from the experimental data set as an input.
This gives us the optimal adaptation considering the stallings times.
In a second step, we remove the cumulative stalling times and force the optimization problem to instantly play the video.

%!TEX root = main.tex
%%%%%%%%%%%%%%%%%%%%%%%%%%%%%%%%%%%%%%%%%%%%%%%%%%%%%%%%%%%%%%%%%%%%%%%%%%%%%%%%

\subsection{Heuristic Approach}
\label{sec:heu}

To describe the heuristic, we first have to define redundant traffic.
The redundant traffic ratio is defined as in the subsequent equation, where $B_T$ is the total amount of data downloaded during the playback session and $B$ is the sum of the segments' sizes shown to the user.

\begin{equation}
	\rho = \frac{B_T-B}{B}
\end{equation}

The heuristic approach uses isotonic regression \cite{barlow1972statistical} to deduce a video-dependent relationship between the data shown to the user and the resulting average quality level based on previously recorded playback sessions.
This gives an estimate of how much non-redundant data is necessary to reach a certain quality level.
Furthermore, it allows us to estimate the difference in terms of average quality between two different amounts of data.
The advantage of the approach is, as previously described, that it captures the dynamics of the overall system as it is based on actual observations.

\begin{figure}[t]
\centering
\includegraphics[width=\columnwidth]{figs/32_vbLLqaa9ksw.pdf}%
\caption{Isotonic regression result showing the relationship between Bytes shown to the user $B$ and resulting average playback quality for video vbLLqaa9ksw. $336$ video views are used in this regression.}
\label{fig:heuristic}%
\end{figure}

Figure \ref{fig:heuristic} illustrates the resulting relationship denoted as function $\phi$ for one of the videos in the data-set.
The axis to the right gives the amount of Bytes shown to the user $B$.
The axis to the left gives the resulting average playback quality.
Each (brown) dot represents one playback session.
The connected (black) dots are the isotonic regression result.
Multiple observations can be made from the figure. 
First, a specific amount of played bytes can result in different average quality levels at the end. 
This is due to the combinatorial problem which arises due to the different quality levels and bit-rate variations inside a quality level.
Second, there is a jump at $\SI{20}{\mega\byte}$ from $0.7$ to $1.1$ average quality level of unknown origin.
Third, there are outliers, e.g. at $\SI{27}{\mega\byte}$, where significant more data does not increase the average quality level.

Based on $\phi$ we determine the loss in average quality level, or \textit{possible gain}, due to the redundant traffic as:

\begin{equation}
\phi(B_T) - \phi(B)
\end{equation}

It is the difference between the average quality level we could have reached with the total Bytes downloaded in the session ($\phi(B_T)$) and the average quality level based on the Bytes shown to the user $\phi(B)$.
%!TEX root = main.tex
%%%%%%%%%%%%%%%%%%%%%%%%%%%%%%%%%%%%%%%%%%%%%%%%%%%%%%%%%%%%%%%%%%%%%%%%%%%%%%%%
 \subsection{Optimization Problem}
%!TEX root = main.tex
%%%%%%%%%%%%%%%%%%%%%%%%%%%%%%%%%%%%%%%%%%%%%%%%%%%%%%%%%%%%%%%%%%%%%%%%%%%%%%%%

\subsection{Data Sets}
\label{sec:datasets}

In total, there are four data sets used in this evaluation as listed in Table \ref{tab:datasets}. The three data-sets starting from the second are calculated based on the first (experimental) data set.
\begin{table}
\caption{Overview of the data sets used in this work.}
\label{tab:datasets}
\centering
\begin{tabular}{p{1.8cm}p{2.1cm}p{3.2cm}}
%\begin{tabular}{|c|c|c|}
\toprule
Data Set & Identifier & Description \\ 
\midrule 
Measurements from \cite{sieber16sacrificing} & measurement & The experimental data set recorded in a testbed. \\ 
\addlinespace
Heuristic estimation & heuristic & The heuristic estimation which gives us the possible gain without redundant traffic. \\ 
\addlinespace
Optimization with stalling & opt (prebuffering) & The MILP solition with stalling times. \\ 
\addlinespace
Opimtization without stalling & opt (instant play) & The MILP solution without stalling times. \\ 
\bottomrule
\end{tabular} 
\end{table}
%Two of those are data sets from \cite{sieber16sacrificing}. The other two were created by implementing above optimization problem in the Gurobi Optimizer

First, we have the initial observations which shall serve as the \textit{baseline} in the following analysis. These measurements were originally recorded in \cite{sieber16sacrificing} where the measurement methodology and measurement set-up is described in more detail: $35$ videos $\times 27$ bandwidth values $\times 15$ replications. Four quality level representations were observed: $144p, 240p, 360p, 480p$. In the following, we refer to these video quality levels as $0,1,2,3$. Please note that stalling events did occur in $56\%$ of these runs.

Based on this data set, we used the \textit{heuristic approach} described in \ref{sec:heu} to estimate the average resolution that is reachable if there was no redundant traffic, i.e. when no video segment is downloaded multiple times. Please note that it was assumed that the same amount of stalling would occur.

As a new contribution, we use the optimization problem, described in Section \ref{optadapt} to exactly calculate the highest mean resolution that was optimally obtainable. As a second step, the number of switches is minimized as first proposed in \cite{miller2013optimal}. For both steps, we limit the execution time of the Gurobi Optimizer to $\SI{1}{\second}$ in order to process the complete data in a timely manner. Increasing the execution will most likely lead to slightly better values than presented in the following. For this \textit{two-step approach}, we consider the same video files, the same duration of the viewing session and the same average network throughput as was used in the baseline scenario to make it comparable. However, instead of having stalling events interrupt the replaying process, we add an initial delay to the replaying process. The duration of this delay is equal to the sum of the observed stalling events. This leads to the same duration of the viewing session and the same replay time and the same amount of data that was totally downloaded. In the following, we refer to this scenario as \textit{opt (prebuffering)}.

Lastly, we present a data set that is obtained in the same fashion as \textit{opt (prebuffering)} with one major difference: the video starts to play immediately after the first segment has been downloaded. To achieve this, we consider the exact same network throughput as in the baseline scenario, while having a shorter session duration since the stalling times are omitted. This means that the amount of data that is downloaded in this case is lower than in the baseline scenario. In the following, we refer to this scenario as \textit{opt (instant play)}.

