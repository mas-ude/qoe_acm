%!TEX root = main.tex
%%%%%%%%%%%%%%%%%%%%%%%%%%%%%%%%%%%%%%%%%%%%%%%%%%%%%%%%%%%%%%%%%%%%%%%%%%%%%%%%

\section{Related Work}
\label{sec:relatedwork}

First, we describe the related work in the area of user perception of HAS video streaming services.
In \cite{hossfeld2015identifying, hossfeld14assessingeffect}, Hoßfeld et al. conclude that avoiding stallings is the first priority when optimizing a HAS service for user experience. 
The second and third priority are the average video quality shown to the user and minimizing the number of switches and the amplitude of the switches.
In \cite{nam16qoe}, Nam et al. conduct a large scale study on YouTube and confirm the high (harmful) impact of re-bufferings and quality switches on the user's QoE.
Further related work in the area of HAS QoE and on HAS in general can be found in \cite{seufert2015survey}.

Next we describe the related work specific to YouTube's adaptation strategy and in particular regarding observed redundant traffic.
In \cite{casas2012youtube}, Casas et al. conclude that the ratio between video bit-rate and downlink bandwidth significantly influences YouTube's adaptation. 
They show that YouTube's adaptation is not robust in bottle-necked scenarios.
Yao et al. show in \cite{Yao2014b} that the iOS YouTube player uses overlapping segments to smoothen the playback.
Rao et al. \cite{rao2011network} and Ito et al. \cite{ito14networklevel} evaluate YouTube's traffic pattern during video playback. They show a dependency of the behavior on the viewing device.
In \cite{Anorga2015}, A\~norga et al. show that YouTube uses a large playout buffer of $\SI{13}{\second}$ to $\SI{40}{\second}$ and therefore can only adapt slowly to changing bandwidth conditions.
In \cite{alcock11application}, Alcock et al. describe YouTube's initial burst phase in detail. They show that $\SI{32}{\second}$ of playback time in a low quality level is transferred to the client as fast as possible before the transfer is throttled. We account this for a major source of redundant traffic as the low quality level is replaced later by higher quality segments.
In \cite{Mansy2014}, Mansy et al. evaluate YouTube's adaptation behavior in terms of redundant traffic, playback behavior and bandwidth utilization. In a wireless scenario with one video and one bandwidth pattern they quantify the redundant traffic to $16\%$. They also show that the adaptation strategies of other content providers behave in a similar way.
Lui et al. \cite{liu2013comparative} conclude that YouTube's buffer level on mobile devices is based on the amount of data buffered, not on the amount of playback seconds. They observe redundant traffic when segments at the beginning of the video are re-downloaded and quantify the redundancy to $15\%$.
In \cite{nam2013mobile}, the authors identify additional redundant traffic on the transport layer of YouTube in a mobile scenario. They quantify the redundant traffic to $35\%$ due to frequent termination of TCP connections and in-flight packets.
