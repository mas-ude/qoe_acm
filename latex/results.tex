%!TEX root = main.tex
%%%%%%%%%%%%%%%%%%%%%%%%%%%%%%%%%%%%%%%%%%%%%%%%%%%%%%%%%%%%%%%%%%%%%%%%%%%%%%%%

\section{Results}
\label{sec:results}

In this section, we present our results on how much YouTube's current adaptation algorithm could be improved. As key metrics, we analyze the average quality, the frequency of stalling events, the frequency of quality switches and the initial delay of the video playback. As a first step, we discuss how the video quality and stalling events are related to each other in the experimental data set.

\subsection{Relationshop Between Quality and Stalling}

\begin{figure}[t]
\centering
\includegraphics[width=\columnwidth]{figs/33qualityvstalling}%
\caption{Average playback quality compared to stalling events per minute as observed in the experimental data set. Average playback quality is clustered using k-means with 40 bins. On average, every 0.6 minutes a stalling event can be observed. Average playback quality and stalling events are highly correlated and show osculating behavior.}
\label{fig:qualityvsstalling}%
\end{figure}

Figure \ref{fig:qualityvsstalling} illustrates the relationship between the average quality level and the stalling events in the experiment result set.
For the average quality level, $0$ is defined as $100\%$ of the segments are shown to the user in $144p$. Quality $3$ is defined as $100\%$ of the segments are shown to the user in $480p$.
The average quality levels are clustered using k-means ($40$ bins) and the error bars indicate the \unit[95]{\%} confidence interval of each cluster.
Two observations can be made from the figure. 
First, the lowest average quality level is $0.3$ with about $0.5$ switches per minute.
From this it follows that the player risks one stalling per two minutes in order to avoid showing only the lowest quality level in low bandwidth scenarios.
Second, the buffering events exhibit an oscillating behavior.
The oscillating behavior is consistent with observations made in \cite{sieber16sacrificing, casas2012youtube}.
The studies show that the performance of YouTube's adaptation algorithm depends on the ratio between video bit-rate and available bandwidth.
For certain ratios, the algorithm is able to efficiently use the available bandwidth, i.e. there is only a low amount of redundant traffic and buffering.
Other ratios exhibit a high amount of redundant traffic and buffering ratio.
In total, the pearson correlation shows a high correlation ($-0.774$) between average quality level and buffering events.
To summarize, in the experimental data, YouTube's adaptation exhibits $0.4$ to $1$ to stalling events per minute.

\subsection{Optimal Adaptation}

Next, we discuss the potential gain in average quality as estimated by the heuristic and the two optimization problem formulations. 

\begin{figure}[t]
\centering
\includegraphics[width=\columnwidth]{figs/qualitygain_py}%
%\caption{Distribution of the mean video quality in the measurement runs and highest achievable mean video quality according to the optimization problem from Section \ref{optadapt} and the heuristic from \cite{sieber16sacrificing}.}
\caption{Distribution of the difference between the observed mean video quality and the  optimally achievable mean video quality  according to the optimization problem from Section \ref{optadapt} and the heuristic from \cite{sieber16sacrificing}.}
\label{fig:opt}%
\end{figure}

Figure \ref{fig:opt} displays the distribution of the difference between the observed mean video quality and the optimally achievable mean video quality. We observe that about $30$ percent of runs are already at maximum quality and can therefore not be improved. The data set \textit{opt (prebuffering)} leads to the highest mean quality. However, the results for the three data sets are very close to each other, e.g. the median of all three data sets is within $0.15$ of a quality of each other.
%In $56$ percent of the baseline runs stalling events did occur.

If we take Figure \ref{fig:initial_delay} into consideration, it becomes clear that this minor difference in quality comes at a price: \textit{opt (instant play)} demonstrates that it would have been possible to avoid stalling and a high initial delay in $93$ percent of cases while increasing the quality in $30$ percent of cases. While \textit{opt (prebuffering)} shows that the mean quality could have been increased by adding an initial delay, the improvement is not particularly high. Lastly, while the heuristic leads to a worse result than \textit{opt (prebuffering)}, it has the advantage of being a less complex problem. This might outweigh the slightly better performance for practical purposes.

\begin{figure}[t]
\centering
\includegraphics[width=1.02\columnwidth]{figs/initial_delay_py}%
\caption{Initial delay for the two optimization data sets}
\label{fig:initial_delay}%
\end{figure}

\begin{figure}[t]
\centering
\includegraphics[width=\columnwidth]{figs/switches_py}%
\caption{Distribution of the number of switches per minute for the heuristic and the optimization}
\label{fig:switches}%
\end{figure}

Finally, Figure \ref{fig:switches} shows the number of quality switches per minute. Here, both data sets that were created with the optimization approach lead to very similar results, which is why we only present the results for \textit{opt (instant play)}. While the number of switches is not of significant importance to the QoE in video streaming according to \cite{seufert2015survey}, continuous video quality switches lead to a low QoE \cite{liu2013user}. The heuristic approach and the optimization both lead to less than $2$ switches per minute in more than $80$ percent of cases which are acceptable values. However, the two-step approach for the optimization leads to some very high switching frequencies that might be problematic. This is because the two-step approach puts very high value on the optimization of the quality level and very little emphasis on the number of switches. Luckily, this problem can easily be averted by using a slightly different approach: In \cite{liotou2016enriching}, a method is proposed that combines both steps into one, allowing the number of switches to be emphasized higher at a negligibly low cost of quality.

