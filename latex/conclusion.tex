%!TEX root = main.tex
%%%%%%%%%%%%%%%%%%%%%%%%%%%%%%%%%%%%%%%%%%%%%%%%%%%%%%%%%%%%%%%%%%%%%%%%%%%%%%%%

\section{Conclusion}
\label{sec:conclusion}

YouTube is a major source of Internet traffic world-wide and it is important to understand how it uses the available resources in a network.
Previous studies revealed that YouTube deploys a user-centric adaptation strategy which allows the player to discard previously downloaded segments and re-download them at a higher quality level.
This increases the average playback quality for the user, but at the same time decreases the overall efficiency.

In this paper we use two methods to quantify this decrease in efficiency.
The first method is a fast heuristic approach based on historical data.
The second method is based on an optimization problem formulation.

Our results show there is still a lot of improvement possible for YouTube. Instead of downloading the same segment multiple times, the wasted traffic could be used to download segments in a higher quality. On average $20\%$ of the videos could have been downloaded in a higher quality. In spite of adaptive mechanisms, stalling usually occurred once per $1$ to $\SI{2}{\minute}$. Assuming the future network bandwidth can be predicted, our optimization problem shows that stalling can be prevented in $94\%$ of these cases. At the same time the initial delay can be kept below $\SI{10}{\second}$ in $95\%$ of cases. In future work, other streaming services such as Amazon Instant Video or Netflix may be investigated with the optimization approach described in this paper.
