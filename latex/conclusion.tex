%!TEX root = main.tex
%%%%%%%%%%%%%%%%%%%%%%%%%%%%%%%%%%%%%%%%%%%%%%%%%%%%%%%%%%%%%%%%%%%%%%%%%%%%%%%%

\section{Conclusion}
\label{sec:conclusion}

YouTube is a major source of Internet traffic world-wide and it is important to understand how it uses the available resources in a network.
Previous studies revealed that YouTube deploys a user-centric adaptation strategy which allows the player to discard previously downloaded segments and re-download them in a higher quality level.
This increases the average playback quality for the user, but in the same time decreases the overall efficiency.

In this paper we use two methods to quantify this decrease in efficiency. 
The first methods is a fast heuristic approach based on historic data.
The second method is based on an optimization problem formulation.






 - The presented methodology can also be applied to other streaming services